% BEFORE CHANGES ARE MADE TO THIS DOCUMENT:
% -References will be automatically updated if any part is added, deleted, etc.
%  However, if a sub part is moved to a different part, its references must be
%  changed.
% -This document must be ratified by the House (as per the Constitution) if
%  changes are to be officialized.

\documentclass{article}
\providecommand{\RevisionInfo}{}
\usepackage{hyperref}
% Reformat section titles
\usepackage{titlesec}

% This package is useful for debugging label problems
% Comment out in final revision
%\usepackage{showkeys}

% Title page information
\title{University Of New Orleans \\ Open Works Collective \\---Circumdans Omnia---}

% Fix margins
\setlength{\evensidemargin}{0in}
\setlength{\oddsidemargin}{0in}
\setlength{\textwidth}{6.5in}
\setlength{\topmargin}{0in}
\setlength{\textheight}{8.5in}

% Use \article for articles and \asection for sections of articles.
% Automatically provide labels with the same article or section title.
\newcommand{\article}[1]{\section{#1} \label{#1}}
\newcommand{\asection}[1]{\subsection{#1} \label{#1}}
\newcommand{\asubsection}[1]{\subsubsection{#1} \label{#1}}
\renewcommand{\thesection}{\Roman{section}}
\renewcommand{\thesubsection}{\arabic{section}.\Alph{subsection}}
\renewcommand{\thesubsubsection}{\arabic{section}.\Alph{subsection}.\arabic{subsubsection}}
\titleformat{\section}{\normalfont\Large\bfseries}{Article \thesection}{1em}{}
\titleformat{\subsection}{\normalfont\large\bfseries}{Section \thesubsection}{1em}{}

% Adding an \asubsubsection -- I feel dirty
%\setcounter{secnumdepth}{5}
\newcommand{\asubsubsection}[1]{\paragraph{#1} \label{#1}}
\renewcommand{\theparagraph}{\arabic{section}.\Alph{subsection}.\arabic{subsubsection}.\Alph{paragraph}}

% Adding \a(sub){3,4}section during merge of bylaws and articles -- I feel _really_ dirty
\setcounter{secnumdepth}{7}
\setcounter{tocdepth}{7}
\newcommand{\asubsubsubsection}[1]{\parindent=0em\subparagraph{#1} \label{#1}}
\renewcommand{\thesubparagraph}{\arabic{section}.\Alph{subsection}.\arabic{subsubsection}.\Alph{paragraph}.\arabic{subparagraph}}

\newcounter{asubsubsubsubsection}[subparagraph]
\renewcommand{\theasubsubsubsubsection}{\arabic{section}.\Alph{subsection}.\arabic{subsubsection}.\Alph{paragraph}.\arabic{subparagraph}.\Alph{asubsubsubsubsection}}
\newcommand{\asubsubsubsubsection}[1]{\parindent=0em\refstepcounter{asubsubsubsubsection}\par\textbf{\theasubsubsubsubsection\hspace{1em}#1 \label{#1}}}

% Headings
\pagestyle{myheadings}
\markright{{\rm UNO OWC Constitution \hfill Page }}

\begin{document}
% Title
\maketitle
\tableofcontents

% ARTICLE I Introduction
\newpage
\article{Introduction}

\asection{Name}
The name of this organization shall be the: 
Open Works Collective;
Abbreviated: OWC

\asection{Purpose}
The purpose of this organization is to garden a productive open ecosystem for communities associated with the University of New Orleans to foster interdisciplinary creativity through the intersection of inclusivity, transparency, and collaboration.

\asection{Objectives}
The objectives of the Open Works Collective are:
\begin{enumerate}
	\item To organize/promote open forms of collaboration with a focus in our university community
	\item To root ourselves in diversity of perspective, thought and interest
	\item To foster mutualistic communitcation across disciplines and departments 
	\item To support the rigorous exploration of creative projects and research
	\item To maintain a common space, digital and otherwise, that is inclusive and accepting to all people
	\item - This space should facilitate community interconnectivity/function/health/growth and be resilient to harmful speech or actions
	\item To develop and offer resources for education relevant to our purpose 
	\item Enable self-determination.
\end{enumerate}

% IDEALS: A potential section regarding our influencing ideals and where they come from - NEEDS REVIEW FROM CEDRIC & RAVEN
%\asection{Ideals}
%Some foundational ideals of the Open Works Collective are:
%\begin{enumerate}
%	\item Openness as derived from the open movement - for the goal of reducing friction in the sharing of information  
%	\item Inclusivity for all 
%	\item Ineropable design - structured for 
%	\item To provide a friendly and comfortable living environment in the residence halls
%\end{enumerate}

%
% ARTICLE II - MEMBERSHIP
%
\newpage
\article{Membership}
There are four major types of membership available to Open Works Collective.
Each carries different qualifications, roles, and privileges.

%\begin{description}
	%\item[Qualifications:] What qualifications an applicant needs to apply
	
	%\item[Expectations:] The duties and responsibilities of Open Works Collective members
	%\item[Privileges:] The benefits offered to House members

	%\item[Resignations:] The process by which a member terminates Open Works Collective membership
	%\item[Term:] The length of time the membership lasts
%\end{description}

\asection{General Membership}
%\asubsection{General Membership Qualifications and Privileges}
\begin{itemize}
	\item General Membership is open to all people officially associated with the University of New Orleans.
	\item The roles of the General members is to act as an ambassadors of the Open Works Collective.
\end{itemize}

\asection{Active Membership}
\begin{itemize}
	\item Active Membership is open to all General members who have attended a meeting within the last two months.
	\item The role of active members is to participate in meetings and projects. 
	\item Active members have the opportunity to become committee members.
\end{itemize}

\asection{Committee Membership}
\begin{itemize}
	\item Committee Membership is open to all Active members.
	\item Committee members have voting rights.
	\item Committee members may participate in more than one Open Works Collective Committee.
	\item The role of committee members is to participate in all their respective committee meetings.
\end{itemize}

\asection{Advisory Membership}
\begin{itemize}
	\item Advisory Membership is open to any University Employee interested in supporting the Open Works Collective efforts.
	\item The Open Works Collective must have at least one advisory member at all times.
	\item Advisory members are not excluded from participating in committees.
\end{itemize}
\newpage
%
% ARTICLE V - Committees
%
\article{Committees}
The Committees are the main governing body of the Open Works Collective.
Its purpose is to provide leadership and direction for the Open Works Collective, to oversee the day-to-day operations of the Collective, and to initiate and organize programs and projects for the Collective.
It is composed of the seven permanent directors and the Chairperson.
\\*\\*
There is one permanent directorship for each major aspect of the government of the Collective and each one is chaired by an Steering Committee member.
Ad Hoc directorships are created on an as-needed basis.
They are generally very task oriented and are chaired by a Collective member.
A directorship has some jurisdiction in its area of interest and often is responsible for the day-to-day decisions regarding its area of interest.
Any large expenditures or large effect decisions must be brought before the entire Collective via the Steering Committee.

\asection{Standing Committees}
A standing committee is a permanent panel that can open or close.
\asubsection{Steering Committee}
\begin{itemize}
	\item The Steering Committee can never be disolved.
	\item The Open Works Collective
	\item 
\end{itemize}

\asection{Ad-Hoc Committees}
An ad-hoc committee is a temporary panel that can open or close, and has the opportunity to become a standing comittee.
\asubsection{Becoming a Standing Committee}
\begin{itemize}
	\item To become a standing committee, an ad-hoc comittee must request standing committee status (by appealing to the steering committee) and a quorum must be reached in favor upon the vote following the request. 
	\item The Open Works Collective
	\item 
\end{itemize}


%\asection{Steering Committee}
%The Steering Committee can never be disolved and is composed of the four officer positions. Shall be res3
%\begin{itemize}
%    \item Shall be responsible for presenting a copy of the minutes to all members of the Faculty/Staff Advisor.
%    \item Shall handle all organization correspondence.
%    \item Shall keep a list of all active members.
%\end{itemize}

%\asubsection{Financial Officers}
%\begin{itemize}
%    \item Shall have the right to vote except when acting as Chair.
%    \item Shall be in charge of all organization finances.
%    \item Shall keep an accurate account of all finances and shall give a report at every meeting.
%    \item Shall be responsible for submitting semester financial reports to the organization.
%\end{itemize}

%\asubsection{Information Officers}
%\begin{itemize}
%    \item Responsible for minutes of officer meetings
%    \item Maintains schedule of events.
%    \item Maintains critical information the club need.
%\end{itemize}

%\asection{Standing committees}
%The Standing Committee is a permanent panel that can open or close.
%\asubsection{External Affairs Committee}
%Shall be responsible for fundraising, public relations, publications, and marketing.
%\asubsubsection{Responsibilities}
%\asubsection{Governance Committee}
%Shall be responsible for the on-going educational oppurtunities for the organization, group recruits, and orientation for new board members.
%\asubsubsection{Responsibilities}
%\asubsection{Internal Affairs Committee}
%Shall be responsible for internal and operational issues related to finance, capital aquistion, and investments for the organization.
%\asubsubsection{Responsibilities}
%\asection{Ad Hoc Committee}

%\renewcommand{\theenumi}{\alph{enumi}} % For this section, we want items to use letters
%\asubsection{Responsibilities of the Steering Committee}
%\begin{enumerate}
%	\item To hold a weekly meeting specific to their responsibilities and submit notes to the House
%	\item To meet, as an Executive Board, at least weekly during the Standard Operating Session, as defined in \ref{Standard Operating Session} to discuss and report the operations of the House
%	\item To report pertinent information to House members at the following House Meeting
%	\item To maintain records of the goals defined by each previous Executive Board
%	\item To act as a Judicial Board as defined in \ref{Judicial}
%	\item To review major projects, as defined in \ref{Expectations of House Members}, presented to them by the Evaluations director
%	\item To make the final vote regarding conditionals and appeals as defined in \ref{Membership Evaluation}
%	\item To respect the privacy of House members confiding in the Executive Board, barring situations related to endangerment of oneself or others, sexual assault, or in the case of a Judicial Board
%	\item To publish a document at the end of each semester to all Members stating House’s accomplishments of that semester
%	\item To review and update the Constitution at the end of each Standard Operating Session, as defined in \ref{Standard Operating Session}.
%		The constitution should remain up to date with current practices.
%\end{enumerate}

%\asubsection{Responsibilities of the Standing Committee}
%\begin{enumerate}
%	\item To preside over Executive Board and House Meetings
%	\item To exercise general supervision over the operations of the Executive Board
%	\item To exercise general supervision over regular House activities
%	\item To act as a liaison to the academic and administrative departments at RIT
%	\item To act as a part of a Judicial Board as defined in \ref{Judicial}
%	\item To cast tie-breaking vote in a split decision in an Executive Board vote
%\end{enumerate}

\asection{Term}
The terms of office for all elected officials shall be two (2) consecutive semesters beginning in the fall.

\newpage
%
% ARTICLE III - VOTING
%
\article{Voting}
This article defines voting procedures.
\asection{Eligibility}
All Open Works Collective members are eligible to vote.
\asection{Definitions}
\asubsection{Total Number of Possible Votes}
The number of Active members eligible to vote.
\asubsection{Total Number of Votes Cast}
The Total Number of Votes Cast is defined as the total number of votes received for every voting option minus the number of abstentions.
\asubsection{Quorum}
A Quorum is defined by the minimum number of votes cast required for a vote to be official.
It is a fraction or a percentage of the total number of possible votes.
Any member present for an Immediate Vote or given a ballot who does not explicitly cast their vote is counted as an Abstention.
A Quorum is reached if the Total Number of Votes Cast plus the number of Abstentions is equal to or exceeding the minimum number of votes required.
\asubsection{Proxy Ballot}
A Proxy Vote is defined as any ballot that was cast by one member on behalf of another member.
Any member may cast a Proxy Vote for another member who is unable to actually participate in the vote.
A Proxy Vote must be explicitly written down and signed by the member not in attendance.
The count of all Proxy Vote must be recorded and announced in all votes.
Proxy Votes are only permissible where explicitly stated, at the discretion of the Chair of the Vote.
\asubsection{Abstention}
An Abstention is defined as a vote indicating a neutral position in the vote.
A means to abstain must always be provided in a vote.
Abstentions are counted towards a Quorum, but not towards the Total Number of Votes Cast used to determine if a vote passes or not.
\asubsection{Vote Counters}
The current members of the Open Works Collective's steering committee are vote counters. There must be at least three committee members to count votes.
%\asubsection{Voting Members}
%\begin{itemize}
 %   \item 1
  %  \item 2
   % \item 3
%\end{itemize}

%\asubsection{Non-Voting Members}
%\begin{itemize}
%	\item Chairperson
%	\item House Secretary
%	\item Ad Hoc Director(s)
%\end{itemize}

\asection{Types of Voting}

\asubsection{Scheduled Vote}
\asubsubsection{Method of Vote}
Votes are cast on paper ballots, which provide a means to indicate every possible option in the vote.
A ballot is then distributed to each acitve Open Works Collective member that is eligible to cast a vote and then votes are collected in the designated ballot box for a pre-specified length of time.
At the end of the voting period, the Chair of the Vote collects the ballots, closing the voting period.
The Vote Counters then tally the results.

\asubsubsection{Voting Period}
For constitutional modification, candidate selection, and officer removal votes, the voting period must be at least forty-eight (48) hours in length.
For any other type of vote, the voting period must be at least twenty-four (24) hours.
The minimum length of the voting period may be explicitly lengthened, but never shortened, in the text describing the actual vote.
\asubsection{Instantaneous Vote}
\asubsubsection{Method of Vote}
The current members of the Open Works Collective's steering committee will state all possible ways to vote, then call out each possibility one at a time.
The chairing member will count the number of members casting their instantaneous vote for that possibility. Instantaneous votes will be planned in advanced and will be held during meetings. The current members of the Open Works Collective's steering committee will decided whether or not the subject being voted on should be instantaneous or scheduled.
\asubsubsection{Voting Period}
An instantaneous vote lasts as long as it takes for all votes to be tallied.

\asection{Number of Votes Required}
The Number of Votes Required refers to the numbers required to achieve a quorum and for a vote to pass.
Below are listed four standard votes.
Numbers for non-standard votes are defined in the section describing the actual vote.
\asubsection{Simple Majority}
In a Simple Majority Vote, a Quorum is reached if the Total Number of Votes Cast is equal to or exceeds one-half the Total Number of Possible Votes.
An option in the vote passes if the number of votes cast for that option is larger than the number of votes cast for every other option individually.
\asubsection{Fifty Percent}
In a Fifty Percent Vote, a Quorum is reached if the Total Number of Votes Cast is equal to or exceeds one-half the Total Number of Possible Votes.
An option in the vote passes if the number of votes cast for that option exceeds fifty percent of the Total Number of Votes Cast.
\asubsection{Two-Thirds}
In a Two-thirds Vote, a Quorum is reached if the Total Number of Votes Cast is equal to or exceeds two-thirds the Total Number of Possible Votes.
An option in the vote passes if the number of votes cast for the option equals or exceeds two-thirds of the Total Number of Votes Cast.



\asection{Ties Between Vote Options}
\asubsection{With Pass/Fail}
If the number of votes cast for the pass option equals the number of votes cast for the fail option, then the vote has failed.
\asubsection{With Multiple Options}
If multiple options may pass, a tie does not present a problem.
If only one option may pass, then the vote must be recast or tabled at the discretion of the current President of the Open Works Collective.
In the event of a tie in a Steering Committee vote, the *Chairperson* may cast the tie-breaking vote.

% ARTICLE IV - AMENDING THIS CONSTITUTION
\article{Amending this Constitution}
The main brainch of the uno-open/governance repository on github is the most upstream version and should be regarded as the OWC's current constitution. 

Collaborative development is done with pull request approach
1 - fork

Any/All merges to it must be reviewed and approved by at least [2 directors/5 members of the governance committee/...]





\asection{Non-Semantic Changes}
There are two methods for non-semantic change to the Constitution.
A Maintainer may approve any proposed change that does not affect the meaning of the document.
Alternatively, the change may be presented to the steering committee for discussion followed by an instantaneous vote.
A quorum of fifty percent of the Total Number of Possible Votes is required for passage.

\asection{Semantic Changes}
Any semantic change to the Constitution requires the change to be proposed in writing for discussion at a Open Works Collective Meeting.
Any modifications made due to the discussion are added to the written proposal and the modified proposal is submitted to the steering committee for evaluation.
The final proposal is presented at the following meeting and ballots are distributed for a scheduled vote.
The ballots are collected for a minimum of a forty-eight hour period.
A quorum of two-thirds of the Total Number of Possible Votes must cast ballots for the vote to be official.
A vote equaling or exceeding two-thirds of the number of votes cast is required for the change to be placed into the constitution.
The Constitution may be overridden by an instantaneous vote under the steering committee.
There must be a quorum of eighty-five percent of the Total Number of Possible Votes for the vote to be official.
A vote equaling or exceeding ninety percent of the number of votes cast is required for the override to take effect.

\asection{Constitution maintainers}

\asubsection{Maintainer Qualifications}
Maintainers must be Active or Alumni in good standing.

\asubsection{Maintainer Expectations}
Maintainers are expected to:
\begin{itemize}
	\item Review changes to the Constitution for grammar, spelling, and internal consistency
	\item Keep a public record of changes to the Constitution
	\item Participate in discussion of proposals
	\item Facilitate the creation of new changes to the Constitution by other members
	\item Be knowledgeable about the Constitution
\end{itemize}
Failure to meet any of these roles is grounds for revocation of Maintainer status by the Executive Board.

\asubsection{Maintainer Selection}
Any member may nominate a qualified member for Maintainer status to the *Executive Board* for consideration.
The *Executive Board* may choose to approve or reject the nomination by Simple Majority vote.
\asection{Impeachment}
An officer may be removed from their position due to negligence of duty, inefficiency within office, and or any other action which is considered detrimental to the name or purpose of the organization. An officer may be removed from office with a scheduled vote and two-thirds of the votes from active members.


\article{Financial Structure}
\asection{Holdings}
Fiscal value will be stored in a uno federal credit union account, the director/chair of the %\ref____ comittee must be on the acct.
%\asubsection{Membership Dues}
%No dues are expected or required to participate in the OWC.
\asubsection{Donations}
Any/all donations recieved will be spent transparently and in the best interest of the community.
%https://opencollective.com/how-it-works
\end{document}
